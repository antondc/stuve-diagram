
\pgfplotsset{select coords between index/.style 2 args={
      x filter/.code={
          \ifnum\coordindex<#1\def\pgfmathresult{}\fi
          \ifnum\coordindex>#2\def\pgfmathresult{}\fi
        }
    }}
\begin{axis}[
    xmin=-95,
    xmax=45,
    ymin=95,
    ymax=1020,
    tick style={draw=none},
    ytick={100,200,...,1000},
    yticklabel style={/pgf/number format/.cd, precision=0, relative*=2},
    y dir=reverse,
    grid=both,
    width=15cm,
    y coord trafo/.code={\pgfmathparse{#1^0.286}},
    y coord inv trafo/.code={\pgfmathparse{#1^(1/0.286)}},
    fixed point arithmetic,
    visualization depends on={value \thisrow{altitude}\as\altitude},
    visualization depends on={value \thisrow{pressure}\as\pressure},
    axis y line*=left,
  ]

  % BASE HEIGHT
  \addplot [black,dashed,select coords between index={0}{0},yshift=2mm] table [x expr={0}, y=pressure, col sep=comma] {data/sounding.csv} node [right] at (axis cs:-95,\pressure) {\pgfmathprintnumber{\altitude} \space m};

  % BASE HEIGHT LINE
  \addplot [black,dashed,select coords between index={0}{0},yshift=-0.3mm,opacity=0.5] table [x expr={0}, y=pressure, col sep=comma] {data/sounding.csv} node {———————————————————————————————————————————————————————————————————};

\end{axis}