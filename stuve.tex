\documentclass[titlepage]{article}
\usepackage{pgfplots}
\usepackage[unicode, hidelinks]{hyperref}
\usepackage[left=3cm, right=3cm, top=3cm, bottom=3cm]{geometry} % Margins
\usepackage{amsfonts}
\pagenumbering{gobble}  % Remove page numbering
\flushbottom
\addtolength{\skip\footins}{1cm}  % Space between main text and footnotes
\usepackage{fp}
\usetikzlibrary{fixedpointarithmetic}

\begin{document}
\begin{tikzpicture}
  \begin{axis}[
      xmin=-95,
      xmax=45,
      ymin=3.668,
      ymax=7.2219,
      y dir=reverse,
      ytick={100^0.2854,200^0.2854,300^0.2854,400^0.2854,500^0.2854,600^0.2854,700^0.2854,800^0.2854,900^0.2854,1000^0.2854},
      yticklabels=\empty,
      xticklabels=\empty,
      xlabel=\(T\),
      ylabel=\(p (hPa)\),
      width=15cm,
    ]
    \input{adiabats.tex}
  \end{axis}
  \begin{axis}[
      xmin=-95,
      xmax=45,
      ymin=95,
      ymax=1020,
      ytick={100,200,...,1000},
      yticklabel style={/pgf/number format/.cd, precision=0, relative*=2},
      y dir=reverse,
      grid=both,
      width=15cm,
      y coord trafo/.code={\pgfmathparse{#1^0.286}},
      y coord inv trafo/.code={\pgfmathparse{#1^(1/0.286)}},
      fixed point arithmetic
    ]
    \input{temperature.tex}
    \input{dew_point.tex}
    \addplot [black,dashed,style={align=left}] coordinates {
    (-95, 898)
    (45, 898)
  } node[pos=0.1,above] { \pgfmathparse{int(round(((10^(log10(898/1013.25)/5.2558797)-1)/(-6.8755856*10^-6))))}\pgfmathresult \space f \\ \pgfmathparse{int(round(((10^(log10(898/1013.25)/5.2558797)-1)/(-6.8755856*10^-6)) * 0.3048))}\pgfmathresult \space m};
  \end{axis}
  \begin{axis}[
      xmin=-95,
      xmax=45,
      ymin=95,
      ymax=1020,
      ytick={100,200,...,1000},
      yticklabel style={align=left,font=\scriptsize},
      yticklabels={
          \pgfmathparse{int(round(((10^(log10(100/1013.25)/5.2558797)-1)/(-6.8755856*10^-6))))}\pgfmathresult \space f \\ \pgfmathparse{int(round(((10^(log10(100/1013.25)/5.2558797)-1)/(-6.8755856*10^-6)) * 0.3048))}\pgfmathresult  m,
          \pgfmathparse{int(round(((10^(log10(200/1013.25)/5.2558797)-1)/(-6.8755856*10^-6))))}\pgfmathresult \space f \\ \pgfmathparse{int(round(((10^(log10(200/1013.25)/5.2558797)-1)/(-6.8755856*10^-6)) * 0.3048))}\pgfmathresult  m,
          \pgfmathparse{int(round(((10^(log10(300/1013.25)/5.2558797)-1)/(-6.8755856*10^-6))))}\pgfmathresult \space f \\ \pgfmathparse{int(round(((10^(log10(300/1013.25)/5.2558797)-1)/(-6.8755856*10^-6)) * 0.3048))}\pgfmathresult  m,
          \pgfmathparse{int(round(((10^(log10(400/1013.25)/5.2558797)-1)/(-6.8755856*10^-6))))}\pgfmathresult \space f \\ \pgfmathparse{int(round(((10^(log10(400/1013.25)/5.2558797)-1)/(-6.8755856*10^-6)) * 0.3048))}\pgfmathresult  m,
          \pgfmathparse{int(round(((10^(log10(500/1013.25)/5.2558797)-1)/(-6.8755856*10^-6))))}\pgfmathresult \space f \\ \pgfmathparse{int(round(((10^(log10(500/1013.25)/5.2558797)-1)/(-6.8755856*10^-6)) * 0.3048))}\pgfmathresult  m,
          \pgfmathparse{int(round(((10^(log10(600/1013.25)/5.2558797)-1)/(-6.8755856*10^-6))))}\pgfmathresult \space f \\ \pgfmathparse{int(round(((10^(log10(600/1013.25)/5.2558797)-1)/(-6.8755856*10^-6)) * 0.3048))}\pgfmathresult  m,
          \pgfmathparse{int(round(((10^(log10(700/1013.25)/5.2558797)-1)/(-6.8755856*10^-6))))}\pgfmathresult \space f \\ \pgfmathparse{int(round(((10^(log10(700/1013.25)/5.2558797)-1)/(-6.8755856*10^-6)) * 0.3048))}\pgfmathresult  m,
          \pgfmathparse{int(round(((10^(log10(800/1013.25)/5.2558797)-1)/(-6.8755856*10^-6))))}\pgfmathresult \space f \\ \pgfmathparse{int(round(((10^(log10(800/1013.25)/5.2558797)-1)/(-6.8755856*10^-6)) * 0.3048))}\pgfmathresult  m,
          \pgfmathparse{int(round(((10^(log10(900/1013.25)/5.2558797)-1)/(-6.8755856*10^-6))))}\pgfmathresult \space f \\ \pgfmathparse{int(round(((10^(log10(900/1013.25)/5.2558797)-1)/(-6.8755856*10^-6)) * 0.3048))}\pgfmathresult  m,
          \pgfmathparse{int(round(((10^(log10(1000/1013.25)/5.2558797)-1)/(-6.8755856*10^-6))))}\pgfmathresult \space f \\ \pgfmathparse{int(round(((10^(log10(1000/1013.25)/5.2558797)-1)/(-6.8755856*10^-6)) * 0.3048))}\pgfmathresult  m,
        },
      y dir=reverse,
      grid=both,
      width=15cm,
      y coord trafo/.code={\pgfmathparse{#1^0.286}},
      y coord inv trafo/.code={\pgfmathparse{#1^(1/0.286)}},
      axis y line*=right,
      fixed point arithmetic
    ]
  \end{axis}
\end{tikzpicture}
\end{document}
